\message{ !name(qwire.tex)}
\message{ !name(qwire.tex) !offset(-2) }
The Qwire language is parameterized by a host language.
In this sense, Qwire really describes a family of languages and not a single one.
However, with only a few assumptions about the host language, namely that it has booleans, product types, and is strongly normalizing \ls{big assumption can’t brush past this easily}, we can give a formal semantics to Qwire.

\subsection{Introduction to Qwire}

Qwire breaks up reasoning about quantum programs into three levels.
At the top, we have the host language capable of everything that a strongly normalizing functional programming language can do.
We require that this host level have boolean and pair types, and we extend the host language with a new type of circuits.
Then we have the circuit layer, which is isolated into a special modality.
It is only in that modality that we can directly manipulate the final level, wires.
Much like in our original introduction of quantum programming, circuits are made up of quantum gates and wires.
Unlike in the quantum lambda calculus, wires and qubits are not standard terms that can be manipulated anywhere in the language.
We cannot create functions from qubits to qubits in Qwire.

\message{ !name(qwire.tex) !offset(-18) }
