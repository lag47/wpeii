\subsection{Advantages of the Quantum Lambda Calculus}
The most impressive aspect of the Quantum Lambda Calculus is how it demonstrates that existing programming languages constructs are enough to capture quantum programming abstractions in full generality.
The abstract view of quantum computation requires us to write algorithms and circuits that abstract over different parameter algorithms or circuits.
This is precisely the kind of abstraction that lambda calculi are well-suited to deal with.
The primary complications that the Quantum Lambda Calculus adds are the use of affine types to control which values are allowed to be reused, and the extension of the operational semantics to handle quantum state and probabilistic effects.
All of this was well understood previously, making it simpler to apply to the application of quantum computing. \todo{this is awkward}

\subsection{Limitations of the Quantum Lambda Calculus}
While it is useful to see how quantum programming can be modelled using with few extensions, the language is not organized along the same lines as the abstractions of quantum computing.
In the Quantum Lambda Calculus, there is no clean distinction between the classical and quantum parts of the any algorithm.
Every function is inherently stateful and probabilistic and the effects are not tracked by the type system at all.

Furthermore, the Quantum Lambda Calculus makes no attempt to separate out static from dynamic lifting.
Classical control flow and controlled quantum gates can be switched between with no constraints.
This is problematic because static lifting is much cheaper to perform then dynamic lifting.
The Quantum Lambda Calculus paper provides no static analysis to distinguish these different kinds of lifting, giving no way to accomplish the computations more efficiently.
In fact it gives no compilation strategy to any lower level quantum computer, making it unclear precisely how to use this language to program a quantum computer rather than just to analyze or classically simulate a quantum computation.

\subsection{Advantages of Qwire}
Qwire makes a clean distinction between the classical and quantum parts of the language.
Classical computations are performed by the host language and quantum computations are performed by the circuit language.
The host language can only pass along boxed circuits as values and not qubits.
Static and dynamic lifting correspond directly to the \inlinecoq{run} and \inlinecoq{lift} keywords directly, making it clear when an actual quantum computer would need to incur the costs and dynamic lifting and when it would not need to.

In \citet{qwirepractice}, Rand and Paykin contributed a verified compiler from Qwire's circuit language to QASM~\cite{qasm}, a popular low level quantum language that can be run directly on some modern quantum computers \todo{check that and edit accordingly}.
This demonstrates that Qwire is not only a useful abstraction for thinking about quantum programming, but can be used as a practical, high level language for quantum computing.

The denotational semantics given to Qwire circuits are closely related to the underlying mathematics of quantum computation, making reasoning about the correctness of Qwire circuits similar to reasoning about the correctness of abstract circuits \todo{confusing terminology}.

The inclusion of dependent types in Qwire's host language allows programmers to fully specify sized quantum circuits, statically ruling out a pernicious class of potential programming errors.

\subsection{Limitations of Qwire}

While dependent types can be useful, in practice they are difficult to work with.
Qwire leaves all details of the dependent types to the host language, giving no extra help to the programmer.
Also, although superoperators over density matrices are a valid mathematical model for quantum circuits, most of the quantum computing resources tend to only consider measurement at the end of a circuit, allowing the circuit to be viewed as a unitary matrix, a simpler mathematical object.
You could imagine separating out Qwire's circuit description language into two parts, one for circuits purely consisting of unitary gates, and one for circuits that include measurement.
The purely unitary circuits could then be denoted as unitary matrices, making them easier to reason about.

\subsection{Related Work}

There are many more quantum programming languages being used. 
Much like QWIRE, Quipper~\cite{quipper} and \liquid~\cite{liquid} both embed circuit descriptions in a functional language, representing circuits as datastructures in the host language.
Quipper is embedded in Haskell, and \liquid is embedded in F\#.
Quipper makes extensive use of Haskell's typeclass.
In Quipper, there is a typeclass \inlinecoq{QData} with instances for types like tuples or lists of qubits.
Quipper also provides a \inlinecoq{Circ} type constructor that is a valid monad.
A quantum circuit with input type \inlinecoq{A} and ouput type \inlinecoq{B} is represented as a function \inlinecoq{A -> Circ B}.
In Quipper, the programmer can compose circuits using the \inlinecoq{bind} operation.
In \liquid, qubits is implemented as an \emph{opaque type} that can only interact with the rest of the environment via a limited interface.
These opaque types also form a monad, although this is not stressed as it is in Quipper, as monads are a bigger part of idiomatic Haskell than they are of idiomatic F\#.
The interface allows for the creation of circuits which can then be passed around like normal values.
Neither Quipper nor \liquid use linear types to prevent cloning, opting instead to use dynamic checks.

The Qiskit~\cite{qiskit}, Cirq~\cite{cirq}, and Q\#~\cite{qsharp} languages are all imperative languages embedded in Python and represent quantum circuits as sequences of imperative operations on qubits.
They are the primary quantum programming languages of IBM, Google, and Microsoft respectively
None of the listed languages use linear types, instead opting either for running static analyses at compile time or throwing exceptions at runtime if a circuit attempts to clone a qubit.
The QASM language~\cite{qasm} is a low level circuit description language that seeks to be an intermediate language representation for higher level languages to target.

There are also many applications of programming languages techniques to quantum programming beyond language design.
In Qwire particularly, there is a body of work on program and program transformation verification~\cite{qwirepractice}.
Researchers also use rich type systems to capture complex, quantum specific program properties.
\citet{twist} designed a type system for keeping track of purity and entanglement, and \citet{gottesmanex} developed Gottesman types as a way to characterize quantum circuits as transformations between eigenvector bases which also allows you to track entanglement.


%Liquid qasm q# q* qiskit do a forward lit review