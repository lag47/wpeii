Like the Quantum Lambda Calculus, Qwire combines features of a circuit description language and a higher order functional language.
Unlike the Quantum Lambda Calculus, Qwire's syntax and type system maintains a clear distinction between quantum circuits, which directly manipulate qubits, and classical terms which behave like normal higher order functional language terms.
The Qwire language is parameterized by a host language.
In this sense, Qwire describes a family of languages and not a single one.

\subsection{Introduction to Qwire}


The QRAM model of quantum computing breaks quantum programs up into two levels, classical computation and quantum circuits.
Qwire maintains the exact same distinction, with the host language handling classical computation and with special circuit syntax to handle quantum circuits.
The host language acts like a standard functional programming language, with a special circuit type.
Qwire allows programmers to create values of the circuit type using the \inlinecoq{box} keyword, under which the programmer can use only the special circuit syntax.
This circuit syntax lets programmers directly manipulate qubit wires, feeding them into quantum gates or boxes.
Wires and qubits are not standard terms that can be manipulated anywhere in Qwire.
This means programmers cannot create host level functions from qubits to qubits in Qwire.

However, programmers can create circuits that are parameterized over input wires acting.
Inhabitants of the type \inlinecoq{Circ(W1,W2)} are like functions from \inlinecoq{W1} to \inlinecoq{W2}, but Qwire's syntax and type system don't allow such types to be used interchangeable.
This helps to maintain Qwire's strict distinction between classical computation and quantum circuits.

With Qwire, we can easily define a closed circuit that creates the entanglement of the states $\ket{00}$ and $\ket{11}$, just as we did in Section~\ref{sec:entangle}.

\begin{lstlisting}[style=customcoq]
x <- gate new0 ();
y <- gate new0 ();
x <- gate hadamard x;
o <- gate control not (x,y);
output o
\end{lstlisting}

\begin{quantikz}
    \lstick{$\ket{0}$} & \gate{H} \slice{$q_1$}& \ctrl{1} \slice{$q2$}& \qw \\
    \lstick{$\ket{0}$} & \qw & \targ{} & \qw
\end{quantikz}\\
\ls{maybe I need a version of this circuit with different annotations}

This circuit code uses the \inlinecoq{gate} keyword to both create new qubits and apply quantum gates.
On the first two lines, we introduce new qubits initialized to the $\ket{0}$ state with the \inlinecoq{new0} gate.
Then we apply the \inlinecoq{hadamard} gate to the \inlinecoq{x} qubit.
Note that we reuse the variable name \inlinecoq{x} with variable shadowing.
We do this to make it obvious which variables correspond to which input wire.

Qwire is parameterized on a collection of base gates. 
Each gate needs to be given a gate type, which specifies the input and output wires of the gate. 
The notion of gate here is broader than the notion of gate in Section~\ref{sec:background}. 
In that section, gates only referred to reversible computations that perform no measurement and preserve superposition information. 
In Qwire these are called unitary gates, while the umbrella term, gate, include measurment and the initial preparation of classical states. 

Given any unitary gate \inlinecoq{u}, we can construct both the the inverse gate \inlinecoq{u}$^\dagger$ and the controlled gate \inlinecoq{control u}.
One such unitary gate is \inlinecoq{not}, so we can construct a controlled version \inlinecoq{not} gate that behaves the same as the CNOT gate in the background section.
Another unitary gate is \inlinecoq{hadamard}.
Given an arbitrary unitary gate \inlinecoq{g}, \inlinecoq{g}$^\dagger$ is the inverse of \inlinecoq{g}, and \inlinecoq{control g} takes a control qubit and a target wire acting as the identity when the control is $\ket{0}$ and acting as \inlinecoq{g} when the control is $\ket{1}$.
Recall from Section~\ref{sec:gates_circs} that this description is sufficient to describe its behavior on arbitrary quantum states.

\subsection{Circuit Host Language Interactions}

There are two ways to embed a circuit into the top level host language.
The first way is to specify its inputs with the \inlinecoq{box} keyword.
\begin{lstlisting}[style=customcoq]
box (x,y) : qubit <@$\otimes$@> qubit
    x <- gate hadamard x;
    o <- gate control not (x,y);
    output o
\end{lstlisting}

The boxed circuit is a term with a circuit type in the host language.
In the above case, the circuit has type Circ($\qubitt\otimes\qubitt,\qubitt\otimes\qubitt$).
Boxed circuits can be \emph{unboxed }and applied to wires inside a different circuit.

The second way to embed circuits into the host language is to take a circuit with no undefined inputs and \emph{running} it with the \run~keyword.
Running a circuit first yields some quantum state, which is then implicitly measured.
For example, consider running the code from above.
\begin{lstlisting}[style=customcoq]
run
    x <- gate new0 ();
    y <- gate new0 ();
    x <- gate hadamard x;
    o <- gate control not (x,y);
    output o
\end{lstlisting}
This program would produce the state $\frac{1}{\sqrt{2}}(\ket{00} + \ket{11})$, an equal superposition, and then measure both wires.
This has an equal likelihood of producing the pair \inlinecoq{(true,true)} and the pair \inlinecoq{(false,false)}.

Recall the distinction between static and dynamic lifting from Section~\ref{sec:lifting}.
The \run~keyword works as a form of static lifting, forcing the entire quantum state to in a circuit to be measured even if we are only interested in a partial output.
For some algorithms, we need dynamic lifting as well, to keep the residual state in superposition while another circuit is chosen based on the measured states.
Qwire enables this with the \inlinecoq{lift} keyword.

Consider the following example program with input wires \inlinecoq{x} and \inlinecoq{y}.
\begin{lstlisting}[style=customcoq]
    b <- lift x;
    y <- unbox (if b then c1 else c2) y;
    output y
\end{lstlisting}
This circuit observes the state in \inlinecoq{x} and uses that value to decide which circuit to apply to \inlinecoq{y}.
This is a form of dynamic lifting.
At first, this might appear similar to the \inlinecoq{control} gate.
However, \inlinecoq{control} only works on unitary gates, where the above program can work with arbitrary circuits.
\todo{need to make this clearer}
The key difference is that \inlinecoq{lift} keyword produce \emph{host language} boolean values, not a circuit level wires, allowing the host language to perform arbitrary classical computation when deciding what circuit to apply to the residual state.
This gives us access to all of the tools of a full programming language to turn this measured value into a circuit.


\begin{figure}
\begin{syntax}
\category[Wire Types]{W_1,...,W_n}
    \alternative{\text{qubit}}
    \alternative{\text{bit}}
    \alternative{W_1\otimes W_2}
\category[Host Types]{\tau_1,...,\tau_n}
    \alternative{...}
    \alternative{\text{bool}}
    \alternative{\tau_1\times\tau_2}
    \alternative{\text{Circ}(W_1,W_2)}
\abstractCategory[Variables]{x,y,z}
\abstractCategory[Base Gates]{bg}
\category[Gates]{g_1,...,g_n}
    \alternative{bg}
    \alternative{bg^\dagger}
    \alternative{\text{control }bg}
\category[Wire Pattern]{w_1,...,w_n}
    \alternative{x}
    \alternative{w_1,w_2}
\category[Circuits]{C_1,...,C_n}
    \alternative{\text{output }x}\\
    \alternative{w_2 \leftarrow \text{gate}~g~w_1;~C}\\
    \alternative{\text{unbox}~C~w}\\
    \alternative{w\leftarrow C_1;~C_2}\\
    \alternative{x\Leftarrow\text{lift}~w;~C}
\category[Terms]{e_1,...,e_n}
    \alternative{...}
    \alternative{\text{box}~(w : W)\Rightarrow C}
    \alternative{\text{run}~C}
\end{syntax}
\caption{Qwire Grammar}
\label{fig:qwiregrammar}
\end{figure}

\subsection{Qwire Type System}
\begin{figure}[t]
    \begin{syntax}
        \category[Wire Context]{\Omega_1, ...,\Omega_n}
            \alternative{\cdot}
            \alternative{\Omega,w : W}
        \category[Host Context]{\Gamma_1, ..., \Gamma_n}
            \alternative{\cdot}
            \alternative{\Gamma, x : \tau}
    \end{syntax}
    \label{qwire_type_cont}
    \caption{Qwire Typing Contexts}
\end{figure}

\begin{figure}[t]
    \rulefiguresize
    \begin{mathpar}
        \inferrule* [Right=Output]
            {\Omega \Rightarrow w : W }
            { \Gamma;\Omega \vdash \text{output}~w }
        \and
        \inferrule* [Right=Gate]
            {\Omega_1 \Rightarrow w_1 : W_1 \and g \in \text{gate}(W_1,W_2) \and \Gamma; w_2 : W_2, \Omega_2 \vdash C : W}
            {\Gamma;\Omega_1, \Omega_2\vdash w_2 \leftarrow \text{gate}~ g ~w_1;~ C : W }  
        \and
        \inferrule*[Right=Unbox]
            {\Omega \Rightarrow w : W_1 \and \Gamma \vdash e : \text{Circ}(W_1,W_2)}
            {\Gamma;\Omega\vdash \text{unbox}~ e~ w : W_2 }  
        \and
        \inferrule*[Right=Compose]
            {\Gamma; \Omega_1 \vdash C_1 : W_1\and \Omega \Rightarrow w : W \and \Gamma;\Omega,\Omega_2 \vdash C_2 : W_2}
            {\Gamma; \Omega_1, \Omega_2 \vdash w \leftarrow C_1;~ C_2 : W_2}
        \and
        \inferrule*[Right=Lift]
            {\Omega_1 \Rightarrow w : W_1 \and \Gamma, x : |W_1|; \Omega_2 \vdash C : W_2}
            {\Gamma;\Omega_1, \Omega_2 \vdash x \Leftarrow \text{lift}~w;~ C : W_2}
        \and
        \inferrule*[Right=BOX]
            {\Omega \Rightarrow p : W_1 \and \Gamma; \Omega \vdash C : W_2}
            {\Gamma \vdash \text{box}~(p: W_1) \Rightarrow C : \text{Circ}(W_1, W_2)}
        \and
        \inferrule*[RIGHT=RUN]
            {\Gamma; \cdot \vdash C : W}
            {\Gamma \vdash \text{run}~ C : |W|}

    \end{mathpar}
        \caption{Typing derivation rules for Qwire circuits and host level terms}
        \label{fig:circuit_typ_der}
    \end{figure}

Qwire uses a mixture of linear and non-linear types in order to prevent violations of the quantum no-cloning theorem while avoiding the cognitive burden of linear types when possible\ls{this sentence could use some work}.
Linear types are used to prevent reuse of wire values inside of circuits.
Non-linear types are used everywhere else in the host language\ls{bad parallelism}.
To maintain this distinction, Qwire has three different typing derivations.
The lowest level derivation is the wire typing derivation, which has the below shape.
\begin{align*}
\Omega \Rightarrow p : W
\end{align*}
$\Omega$ refers to a context of wire variables.
Intuitively, $p$ is some arrangement of the wire variables in $\Omega$ with type $W$.
You can also read this derivation as saying that $\Omega$ is a collection of wires that have been permuted into value $p$ with type $W$.
All of the inference rules for this derivation are given in Figure~\ref{fig:wire_type_der}.
These typing rules preserve the invariant that every variable bound in $\Omega$ is used in $p$ exactly once.

At the next level is the circuit typing derivation which has the below shape.
\begin{align*}
\Gamma;\Omega \vdash C : W
\end{align*}

$\Omega$, once again, refers to the linear context of wire variables.
$\Gamma$ refers to the nonlinear context of host language variables.
So while $\Omega$ contains exclusively tensor products of bits and qubits, $\Gamma$ can contain data structures like natural numbers, functions, and circuits.
And while variables in $\Omega$ must all be used exactly once, variables in $\Gamma$ can be used arbitrarily many times.
In the rest of this section, we will introduce and explain each typing rule for Qwire.
All of the typing rules are collected in Figure~\ref{fig:circuit_typ_der}.

\todo{it might be better to inline the typing rules here}

The OUTPUT rule provides a way to terminate a circuit with some collection of wires.
Note that this rule provides no direct way have values bound in $\Gamma$ influence $p$.

\begin{mathpar}
\inferrule* [Right=Output]
    {\Omega \Rightarrow w : W }
    { \Gamma;\Omega \vdash \text{output}~w }
\end{mathpar}

The GATE rule ensures that inputs consumed by a gate cannot be used in the resulting circuit, unless of course the gate leaves those wires unchanged.
It accomplishes this by having the typing assumption for the continuation $C$ replace the context with the single pattern output by the gate.
So none of the inputs to the gate are in scope when typing the continuation.

\begin{mathpar}
\inferrule* [Right=Gate]
    {\Omega_1 \Rightarrow w_1 : W_1 \and g \in \text{gate}(W_1,W_2) \and \Gamma; w_2 : W_2, \Omega_2 \vdash C : W}
    {\Gamma;\Omega_1, \Omega_2\vdash w_2 \leftarrow \text{gate}~ g ~w_1;~ C : W }  

\end{mathpar}

The UNBOX rule proves one way to have the host context $\Gamma$ influence a circuit.
The UNBOX rule requires a circuit $\coqfont{c}$ to be typed in the host typing derivation.
Unlike wire values, this circuit could come as the input of some host level function or be the direct result of some complex host level computation.
It also requires and consumes an input pattern, preventing wire variables from being forgotten or reused.

\begin{mathpar}
\inferrule*[Right=Unbox]
    {\Omega \Rightarrow w : W_1 \and \Gamma \vdash e : \text{Circ}(W_1,W_2)}
    {\Gamma;\Omega\vdash \text{unbox}~ e~ w : W_2 }  

\end{mathpar}

The LIFT rule provides another, more subtle, way for $\Gamma$ to influence a circuit.
As previously mentioned the \coqfont{lift} keyword is used for dynamic lifting. \ls{not sure the reader remembers what that is}
Note that unlike the GATE rule, the LIFT rule binds a variable by adding it to the host context $\Gamma$.
It uses the \coqfont{|$\cdot$|} function to map wire types onto products of booleans.
This allows us to use host language control flow operators on this value, selecting between one of many different circuits to apply to the remaining quantum wires in $\Omega_2$.
\begin{mathpar}
\inferrule*[Right=Lift]
    {\Omega_1 \Rightarrow w : W_1 \and \Gamma, x : |W_1|; \Omega_2 \vdash C : W_2}
    {\Gamma;\Omega_1, \Omega_2 \vdash x \Leftarrow \text{lift}~w;~ C : W_2}
\end{mathpar}

The COMPOSE rule lets you take the output wires of some circuit and use it in defining some continuation circuit.
This is particularly useful in connecting unboxed circuits with gates.

\begin{mathpar}
\inferrule*[Right=Compose]
    {\Gamma; \Omega_1 \vdash C_1 : W_1\and \Omega \Rightarrow w : W \and \Gamma;\Omega,\Omega_2 \vdash C_2 : W_2}
    {\Gamma; \Omega_1, \Omega_2 \vdash w \leftarrow C_1;~ C_2 : W_2}
\end{mathpar}

Finally, we have the host level typing derivation, with the below shape.
\begin{align*}
\Gamma \vdash e : \tau
\end{align*}

Most of this typing derivation is going to be dependent on the host language.
There are only two rules that are specific to Qwire.
First we have the BOX rule, which allows us to take an open circuit, give a type for its input, and store it as a host term.
\begin{mathpar}
\inferrule*[Right=BOX]
    {\Omega \Rightarrow p : W_1 \and \Gamma; \Omega \vdash C : W_2}
    {\Gamma \vdash \coqfont{box (p}: W_1) \Rightarrow C : \coqfont{Circ}(W_1, W_2)}
\end{mathpar}

Second we have the RUN rule, which allows us to take a circuit with no inputs, execute it and return the measured results.
\begin{mathpar}
\inferrule*[RIGHT=RUN]
    {\Gamma; \cdot \vdash C : W}
    {\Gamma \vdash \coqfont{run } C : |W|}
\end{mathpar}
This typing rule forces the argument of \coqfont{run} to have no inputs by forcing it to be typed with an empty wire context.

\begin{figure}[t]
    \rulefiguresize
    \begin{mathpar}
        \inferrule*[]{ }{ \emptyc \Rightarrow \unitv : \unitt}
        \and
        \inferrule*[]{ }{x : W \Rightarrow x : W}  
        \and
        \inferrule*[]
            {\Omega_1 \Rightarrow p_1 : W_1 \and \Omega_2 \Rightarrow p_2 : W_2}
            {\Omega_1,\Omega_2 \Rightarrow (p_1,p_2) : W_1 \otimes W_2}  
    \end{mathpar}
    \caption{Inference rules for wire typing derivation}
    \label{fig:wire_type_der}
\end{figure}



\subsection{Functional Entanglement Implementation}

\paragraph{Dependent Types}
In this presentation we assume that the host language has dependent types.
In the background section we talked extensively about $n$ by $n$ quantum circuits for abstract $n$.
Dependent types allow us to parameterize the types of wires and circuits with sizes.
This allows us to fully formalize the example circuit in full generality.

To begin, we can write the following host level function from positive integers to wire types that implements sized qubit vectors.
\begin{lstlisting}[style=customcoq]
Fixpoint qubits (n : pos) : WireType :=
    match n with
    | 1 => qubit
    | S m => qubit <@$\otimes$@> (qubits m)
    end. 
\end{lstlisting}

Then we can formalize the extended hadarmard circuit, as well as an n qubit ancillary 0 state.
\begin{lstlisting}[style=customcoq]
Fixpoint extended_had (n : pos) : Circ (qubits n, qubits n) :=
    match n with
    | 1 => box (x : qubit) => x <- gate hadamard x; output x
    | S m => box ((x, y) : (qubit <@$\otimes$@> qubits m)) => 
                x <- gate hadamard x;
                y <- unbox (extended_had m) y;
                output (x,y)
    end.

Fixpoint extended_zero (n : pos) : Circ (<@\unitt@>, qubits n) :=
    match n with
    | 1 => box (_ : <@\unitt@>) => x <- gate new0 (); output x
    | S m => box (_ : <@\unitt@>) =>
                x <- gate new0 ();
                y <- unbox extend_zero m;
                output (x,y)
    end.
\end{lstlisting}


With these programs written, we can write a function that takes in a size $n$, a $2n$ by $2n$ circuit that we assume is the quantum oracle of some function, and produce the functional entanglement circuit.
\begin{lstlisting}[style=customcoq]
Definition functional_entanglement (n : pos) 
    (Qf : Circ(qubits (2 * n), qubits (2 * n) )) : Circ (<@\unitt@>, qubits (2 * n) ) :=
        box (_ : <@\unitt@>) =>
            x <- unbox extended_zero n;
            y <- unbox extended_zero n;
            x <- unbox (extended_had n) x 
            unbox Qf (x,y).
\end{lstlisting}

This program precisely matches the circuit description.
It is important to note that while this section is written mostly in Coq syntax for the host language, actually using Coq's dependent types for this purpose would be more complicated than it appears here.

%here I am not entirely sure, maybe bring it up with Steve I feel like 
%explaining every rule for the operational semantics
% I would argue I am more interested in the types, so I can gloss over the specific inference
% rules
\subsection{Qwire Operational Semantics}
Qwire has an operational semantics based on the operational semantics for the host language.
Given the relations $\hoststep$, which is a small step semantics for host terms, and $\boxstep$, which is a small step semantics for boxed circuits,
we can define $\Rightarrow$, the small step semantics for circuits.
We also assume that the host language has some way of simulating quantum circuits in order to simulate the run keyword.
\ls{here I can point a pointer to a figure that just includes all of the rules}

Let us assume that $\hoststep$ satisfies progress, preservation and strong normalization.
Then we can prove the corresponding properties of Qwire.
In these theorem statements, the variable $\mathscr{Q}$ ranges over wire contexts where every variable either has type qubit or bit.
An open circuit $C$, where $\cdot;\mathscr{Q}\vdash C : W$, is normal if it consists only of output statements, gate applications, and dynamic lift statements.
As the name normal suggests, such circuits cannot reduce any further.

\begin{theorem}[Preservation]
\thmbegin
\begin{enumerate}
    \item If $\cdot \vdash t : \tau$ and $t \rightarrow t'$, then $\cdot \vdash t' : \tau$.
    \item If $\cdot; \mathscr{Q}\vdash C : W$ and $C \Rightarrow C'$ then $\cdot; \mathscr{Q}\vdash C' : W$.
\end{enumerate}
\end{theorem}

\begin{theorem}[Progress]
\thmbegin
\begin{enumerate}
    \item If $\cdot \vdash t : \tau$, then either there exists some term $t'$ such that $t \rightarrow t'$, or $t$ is a value.
    \item If $\cdot;\mathscr{Q}\vdash C : W$ then either $C$ is normal or there exists some term $C'$ such that $C \Rightarrow C'$.
\end{enumerate}
\end{theorem}

\begin{theorem}[Strong Normalization]
\thmbegin
\begin{enumerate}
    \item If $\cdot \vdash t : \tau$ then there exists some value $v$ such that $t \rightarrow^* v$.
    \item If $\cdot; \mathscr{Q}\vdash C : W$ then there exists some normal circuit $N$ such that $C \Rightarrow^* N$.
\end{enumerate}
\end{theorem}
\subsection{Qwire Denotational Semantics}
While progress, preservation, and strong normalization are all very useful properties for programming languages, they don't directly relate to some of the core issues of quantum programming.
In particular, these theorems do not neccessarily imply that inhabitants of circuit types are implementable even by idealized quantum computers.
We could have relaxed the constraints of the linear types, and ended up with a language that could have been useful for reasoning about classical circuits that has those properties as well.

Qwire also comes with a denotational semantics specifically for circuits.
This denotational semantics assigns every well typed circuit to matrices that map density matrices over the input wire denotation to density matrices over the output wire denotation.
This wire type denotation is given below.
\begin{align*}
[bit] = [qubit] &= \mathbb{C}^2\\
[W_1 \otimes W_2] &= [W_1] \otimes [W_2]\\
[\unitt] &= \mathbb{C}
\end{align*}
Take as a given that all of the above vector spaces are restricted to unit norm vectors.
%here I am actually kind of confused, the paper says circuits are operators between density matrices
%but shouldn't that be the same as operators between hilbert spaces
%maybe no because of meas and new
% every operator between hilbert spaces can be lifted to an operator between density matrices
% but the other way around does not hold
% at least I am pretty confident my misundertanding lies in this part and not the other
We can denote $[\cdot; \Omega \vdash C : W ]$ as a superoperator from $[\Omega]^*$ to $[W]^*$.
At a base level, this requires denoting every gate as a superoperator as well.
For unitary matrices, this is done as normal.\todo{fill in more when I understand more }
Here are some of the denotations of other mentioned gates \ls{bad sentence}.
\begin{align*}
[init0] &= (\ket{0}\ket{0}^\dagger)^*\\
[init1] &= (\ket{1}\ket{1}^\dagger)^*\\
[meas]  &= (\ket{0}\ket{0}^\dagger)^* + (\ket{1}\ket{1}^\dagger)^*
\end{align*}

With all of these pieces in place, the denotational semantics satisfies the following theorem.
\begin{theorem}
If $\cdot;\mathscr{Q}\vdash C : W$ and $C \Rightarrow C'$ then 
$[\cdot;\mathscr{Q}\vdash C : W] = \cdot;\mathscr{Q}\vdash C' : W$.
\end{theorem}

This theorem assures us that, assuming all of the given gates are realizable by our quantum computer, then every circuit specified in a Qwire program is realizable on our idealized quantum computer.